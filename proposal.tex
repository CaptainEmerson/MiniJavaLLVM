\documentclass[11pt]{article}
\usepackage{acronym}
\usepackage{titling}
\usepackage[small,compact]{titlesec}
\usepackage{enumitem}
\usepackage[margin=1.25in]{geometry}

\pagenumbering{gobble}

\setlist{nolistsep}

\title{MiniJava LLVM Proposal}
\author{Mitch Souders and Mark Smith}
\date{May 2, 2014}
\begin{document}

\makeatletter
\def\maketitle{\par{\centerline{\huge\bfseries\@title}}\par\@author -- \@date}
\makeatother

\maketitle

\section*{Project Topic}
The primary goal of this project is to gain familiarity with \texttt{LLVM}. The proposal is to create an implementation for compiling MiniJava to the \texttt{LLVM} runtime IR.  The basis for the starting framework will be the MiniJava application used in CS321/322.  The compiler starter framework was provided by Professor Mark Jones.  The MiniJava project currently includes options for running as an interpreter or compiling to x86 code.

The project proposal is to extend the back-end for these existing constructs to emit \texttt{LLVM} IR:

\begin{itemize} 
\item classes
\item class fields
\item class methods
\item variable declarations
\item this invocation
\item object access
\item statements and blocks
\item do-while loops
\item if-then constructs.  
\end{itemize}

\section*{Extensions}
The project proposal is to also extend the existing functionality by adding additional features to the MiniJava Compiler (MJC).  These would be added in the the following order:

\begin{itemize}
\item Garbage Collection
\item Dynamic Dispatch
%\item Static Methods/Fields   % This appears to already exist in the MJC.
%\item Object Fields           % Already Exists
\item Extensions to MiniJava (e.g. Interfaces)
\end{itemize}


\section*{Deliverables}
The project deliverables will be a summary of the work, any learning during implementation, the completed compiler as Java source code and unit tests indicating correct operations for language construct snippets.


\section*{Materials}
The following tools will be used to implement this project. Note: Some of these tools have alternates that could be used if the specified tool prove unsuitable (e.g. using \texttt{llvm-j} or \texttt{java\_cc} instead of \texttt{jllvm}).

\begin{description}
\item[MiniJava Compiler]
	The MiniJava Compiler (\texttt{MJC}) from CS321/322 will be considered the base version of the language MiniJava with further extensions added as required.

\item[LLVM]\footnote{http://llvm.org/} The target compiler for the generated IR for MiniJava.

\item[jllvm]\footnote{https://code.google.com/p/jllvm}
Java bindings to \texttt{LLVM}, which allow direct interaction with \texttt{LLVM}.

\end{description}

\end{document}
