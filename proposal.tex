\documentclass[11pt]{article}
\usepackage{acronym}
\usepackage{titling}
\usepackage[small,compact]{titlesec}
\usepackage{enumitem}
\usepackage[margin=1.25in]{geometry}

\pagenumbering{gobble}

\setlist{nolistsep}

\title{MiniJava LLVM Front-end Proposal}
\author{Mitch Souders}
\date{April 25, 2014}
\begin{document}

\makeatletter
\def\maketitle{\par{\centerline{\huge\bfseries\@title}}\par\@author -- \@date}
\makeatother

\maketitle

\section*{Project Topic}
The primary goal of this project is to gain familiarity with \texttt{LLVM} by creating a front-end for \texttt{MiniJava} for \texttt{LLVM}. 

\section*{Approach}
This will require writing an application in Java using the \texttt{jllvm} Java bindings to \texttt{LLVM} to produce the IR that is suitable for execution/compilation by \texttt{LLVM}. 

Note: The website for MiniJava does provide skeletons for many of the classes for MiniJava implementation with the logic left unimplemented.

The proposed flow is:
\begin{enumerate}
\item Lexing/Parsing of MiniJava with \texttt{jcup} creating an Abstract Syntax Tree.
\item Semantic Verification such as variable checks and type checking
\item Traversing AST using \texttt{jllvm} bindings to produce \texttt{LLVM} IR.
\end{enumerate}

\section*{Possible Extensions}
If this project does not meet the scope or is completed quickly, it may be necessary to add additional requirements. 

\begin{itemize}
\item Extensions to MiniJava (e.g. Interfaces)
\item Garbage Collection
\end{itemize}

\section*{Deliverables}
A working, demonstrable implementation of \texttt{MiniJava} using the \texttt{LLVM} front-end, along with all source code. A project progress report will be produced midway through the project. Additionally a project summary will also be provided, which will include implementation details, overcome issues, limitations of the \texttt{MiniJava} front-end and \texttt{LLVM}.

\section*{Materials}
The following tools will be used to implement this project. Note: Some of these tools have alternates that could be used if the specified tool prove unsuitable (e.g. using \texttt{ANTLR} or \texttt{java\_cc} instead of \texttt{jcup}).

\begin{description}
\item[MiniJava]\footnote{http://www.cambridge.org/us/features/052182060X/}
	The desired front-end language to convert into \texttt{LLVM} IR. The grammar listed on the website will be assumed to be the canonical version of the language to be implemented.

\item[LLVM]\footnote{http://llvm.org/} The target compiler for the generated IR for MiniJava.

\item[jllvm]\footnote{https://code.google.com/p/jllvm}
Java bindings to \texttt{LLVM}, which allow direct interaction with \texttt{LLVM}.

\item[jcup]\footnote{http://www2.cs.tum.edu/projects/cup/}
Java parser generator necessary to build AST for MiniJava.

\end{description}

\end{document}
